\chapter{Unstructured Notes}

\section{How are differential operators related to semi group}
Sobolev inequalities relate norms of Sobolev spaces and can be used to prove embedding of $W^{k,p}(\Real^n) \in W^{l,q}(\Real^n)$ Logarithmic Sobolev inequalities.  The Rellich–Kondrachov theorem makes this precise for the $L^p$ spaces.
$A$ is an operator on a $L^p$ Banach space. We define this in an operator sense as the sum given by the Taylor expansion of $e^x = \sum \frac{x^n}{n!}$
\begin{equation*}
  e^{-t A} : L^p \rightarrow L^q
\end{equation*} is contractive if this holds for $p=2$ and $q=4$.
$e^{-t A}$ is a positive preserving contractive semi-group if
\begin{eqnarray*}
  e^{-t A} f &\geq& 0 \\
  \parallel e^{-t A} f \parallel_p &\leq& \parallel f \parallel_p
\end{eqnarray*}
When $e^{-t A}$ maps $L^2$ to $L^\infty$ we call the semi group untracontractive.

Covering number

Union bound

Kashin representation

Candes and Tao extend Uncertainty Principle to discrete domain
For $x \in \Complex^n$
\begin{equation*}
  |supp(x)| |supp(\hat{x}) \geq N
\end{equation*}
Uniform uncertainty principle is related to the restricted isometry property.

The $\Lambda$ problem of Talingrad relates Fourier matrix to HUP question.  Consider matrices which are euclidian projection of cube $Q^N = {x : \parallel x \parallel_\infty  \leq 1}$, a problem form geometric functional analysis.  Let $B^n = {x : \parallel x \parallel_2  \leq 1}$

Kashin; $\exists$ orthogonal projection $P$
\begin{equation*}
  P : Q^N \rightarrow U \in \Complex^N \st U \approx B^n
\end{equation*}
This implies
\begin{equation*}
  \exists A \in M^{n,N} \st B^n \in \frac{K}{\sqrt{N}} A Q^N \in K B^n where K = K(\lambda) \lambda=\frac{N}{n}
\end{equation*}
$\lambda$ is the redundancy.  This statement is essentially telling us that we can project the cube into the ball when N and n allow for it.

