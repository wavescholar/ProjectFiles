\chapter{Appendix Statistics}


\section{Testing for normality and other distributions }
Powerful inference methods can be employed when data is generated by a Gaussian process. This section describes techniques for testing the normality of a sample and comparing two samples.

Kolmogorov-Smirnov test uses the fact that the empirical
cumulative distribution function is normal in the limit. It is
a non-parametric and distribution free test. Given the
empirical distribution \[F_n(x) = \frac{1}{n}
\sum\limits_{n}^{i=1} \biggl\{\begin{array}{c}1 :x_i\leq x \\0
: x_i>x\\ \end{array}\], and a test CDF\[ F(x)\] the K-S test
statistics are $D_n^+ = max(F_n(x)-F(x) ) $ and $D_n^- =
min(F_n(x)-F(x) ) $ The generality of this test comes at a loss
in precision near the tails of a distribution.  The K-S
statistics are more sensitive near points close to the median,
and are only valid for continuous distributions.

The Kuipers test uses the statistic $D_n^+ + D_n^- $ and is
useful for detecting changes in time series since the statistic
is invariant in ???? transformation of the dependent variable $
F_n$.

The Anderson-Darling test is based on the K-S test and uses the
specific distribution to specify the ????critical values??? of
the test.

The chi-squared is based on the sample histogram and allows
comparison against a discrete distribution, but has the
potential drawback of being sensitive to how the histogram is
binned and requires more samples to be valid.

The Shapiro-Wilk test uses the expected values of the order
statistics of $F(x)$ to calculate the test statistic.  It is
sensitive to data that are very close together, and numerical
implementations may suffer from a loss of accuracy for large
sample sizes.

K-S [Chakravarti, Laha, and Roy, (1967). Handbook of Methods of
Applied Statistics, Volume I, John Wiley and Sons, pp.
392-394].

Shapiro-Wilk [Shapiro, S. S. and Wilk, M. B. (1965). "An
analysis of variance test for normality (complete samples)",
Biometrika, 52, 3 and 4, pages 591-611.]

D’Agostino-Pearson


\section{Regression Methods}
%The multiple regression model in matrix notation can be expressed
%as
%
%revisit - These equations hold for the univariate and the
%multivariate case.
%
%$\textbf{Y} = \textbf{X}{\beta} + \textbf{e}$. %%%%%%%%%%%%%555555
%
%where $\textbf{e} =_d N(\mathbf{\mu},mathbf{\Sigma})$
%
%$E \textbf{Y} = mathbf{\mu} X mathbf{\beta}$
%
%$var \textbf{Y} = \sigma^2 \textbf{I}$
%
%Maximum Likelihood and Least Squares give same estimator;
%
%$\widehat{mathbf{\beta}}=(X^TX)^{-1} X^T Y$
%
%$\hat{\sigma}^2 = frac{1}{n} || Y- \hat{\mu} ||^2$
%
%Multivariate regression is the extension to YM = Xb + e
%
%(called GLM)
%
%Here Y, X, b, and e are as described for the multivariate
%regression model and M is an m x s matrix of coefficients defining
%s linear transformation of the dependent variables. The normal
%equations are
%
%X'Xb = X'YM
%
%and a solution for the normal equations is given by
%
%b = (X'X)-X'YM
%
%Here the inverse of X'X is a generalized inverse if X'X contains
%redundant columns.
%
%Add a provision for analyzing linear combinations of multiple
%dependent variables, add a method for dealing with redundant
%predictor variables and recoded categorical predictor variables,
%and the major limitations of multiple regression are overcome by
%the general linear model. To index
%
%(GLZ) Generalized Linear Model When attempting to explain
%variation in more than one response variable simultaneously the
%modeling exercise is to fit the General Linear Multivariate Model
%(GLMM) to the data. Commonly used multivariate statistical
%procedures such as Multivariate Analysis of Variance (MANOVA),
%Multivariate Analysis of Covariance (MANCOVA), Discriminant
%Function Analysis (DFA), Canonical Correlation Analysis (CCA), and
%Principal Components Analysis (PCA) are all forms of the GLMM.
%When the distribution of the response variable(s) is not normal or
%multivariate normal, or if the variances or the
%variance-covariance matrices are not homogeneous, then application
%of hypothesis tests to GLUM’s or GLMM’s can lead to Type I and
%Type II error rates that differ from the nominal rates.
%Traditionally, transformations of the scale of the response
%variables have been applied to insure that the assumptions
%required for hypotheses tests are met. For example, count data are
%often Poisson distributed and tend to be right skewed.
%Furthermore, the variance of a Poisson random variable is equal to
%the mean of the response. Hence, for count data a transformation
%must both normalize the data and eliminate the inherent variance
%heterogeneity. Commonly, count data are transformed to a
%logarithmic scale or even a square-root scale, however such
%transformations are not always successful in achieving the desired
%end. In fact, there is no a priori reason to believe that a scale
%exists that will insure that data meet the normality and variance
%homogeneity assumptions.The Generalized Linear Model is an
%extension of the General Linear Model to include response
%variables that follow any probability distribution in the
%exponential family of distributions. The exponential family
%includes such useful distributions as the Normal, Binomial,
%Poisson, Multinomial, Gamma, Negative Binomial, and others.
%Hypothesis tests applied to the Generalized Linear Model do not
%require normality of the response variable, nor do they require
%homogeneity of variances. Hence, Generalized Linear Models can be
%used when response variables follow distributions other than the
%Normal distribution, and when variances are not constant. For
%example, count data would be appropriately analyzed as a Poisson
%random variable within the context of the Generalized Linear
%Model. Parameter estimates are obtained using the principle of
%maximum likelihood; therefore hypothesis tests are based on
%comparisons of likelihoods or the deviances of nested models. The
%common linear regression model (a form of the general linear
%model) specifies that the mean response µ is identical to a linear
%function ? of the predictor variables xj:of the predictor
%variables xj:
%
%and uses least squares as the criterion by which to estimate the
%unknown parameters ß   = (ß0,        ß1,...,  ßp)'. When
%observations are independent and normally distributed with
%constant variance s2, least squares estimation of ß   and s2 is
%equivalent to maximum likelihood estimation.
%
%Generalized linear models encompass the general linear model and
%enlarge the class of linear least-squares models in two ways: the
%distribution of Y for fixed x is merely assumed to be from the
%exponential family of distributions, which includes important
%distributions such as the binomial, Poisson, exponential, and
%gamma distributions, in addition to the normal distribution. Also,
%the relationship between E(Y) = \mu and ? is specified by a
%non-linear link function ? = g(µ), which is only required to be
%monotonic and differentiable. The link function serves to link the
%random or stochastic component of the model, the probability
%distribution of the response variable, to the systematic component
%of the model (the linear predictor):
%
% Where g(µ) is a non-linear
%link function that links the random component, E(Y), to the
%systematic component .  For traditional linear models in which the
%random component consists of the assumption that the response
%variable follows the Normal distribution, the canonical link
%function is the identity link. The identity link specifies that
%the expected mean of the response variable is identical to the
%linear predictor, rather than to a non-linear function of the
%linear predictor
%

Standard least squares regression consists in fitting a line
through the data points (training points in learning theory)
that minimizes the sum of square residuals.  The underlying
assumption is that the data and the response can be modeled by
a linear relationship.  In the event that the model accurately
captures the functional dependence of the response generated by
the data, and under the assumptions that the data is corrupted
by Gaussian noise, precise statistical inferences can be made
on the model parameters.

Modifications to this standard model include nonlinear mapping
of the input data, local fitting, biased estimators, subset
selection, coefficient shrinking, weighted least squares, and
basis expansion transformations.

\section{Generalized Linear Models}
Suppose we have $n$ observations of $k$ dimensional data
denoted $\{x_i\}_{i=1}^{k}$ and for each observation we have a
response $y_i$. We wish to fit the observations to the
responses. Generalized Linear Regression is a modeling
technique that allows for non normal distributions and models
non-linear relationships in the training data. M-estimators are
used to fit a generalized linear model Ref Huber (1964).

A linear model $ Y =\Lambda(X)=X\beta + \epsilon$ fits a linear
relationship between the dependent variables $Y_i$ and the
predictor variables $X_i$
\begin{equation}Y_i=\Lambda(X_i)=b_o + b \circ X_i.\end{equation}A generalized
linear model $Y= g(\Lambda(X) ) + \epsilon $ fits the data to $
Y = g (X \circ W)$. Fitting the model consists of minimizing
the objective function $\sum\limits_{i=1}^{n}
g(e_i)=\sum\limits_{i=1}^{n} g(y_i- x_i \beta)$, where $e_i$
are the residuals $y_i-x_i \beta$. We see that for ordinary
least squares $g(e_i)=e_i^2$, and the usual matrix equations
fall out by differentiating with respect to $\beta$. Carrying
this out for general $g$
\begin{equation}
\sum\limits_{i=1}^{n} \frac{\partial  g(y_i-x_i \beta)}{\partial
\beta}=0
\end{equation}
gives the system of $k+1$ equations to solve for estimating the
coefficients $b_i$.  If we set$\alpha(x)=\frac{g'(x)}{x}$ and
calculate the derivative above, we have to solve
\begin{equation}
\sum\limits_{i=1}^{n} \omega(e_i) (y_i-x_i \beta) x_i = 0
\end{equation}
%bbcevisit cast in matrix notation as well.
Which gives rise to a weighted least squares where the weights
depend on the residuals - which depend on the coefficients -
which depend on the weights.  This suggests an iterative
algorithm;
\begin{equation}
\beta^\tau = ( X^{t} W^{(\tau-1)} X )^{-1} X^{t} W^{\tau-1} y
\end{equation}
where $W_{ij}^{(\tau-1)}=\alpha(e_{i}^{(\tau-1})$.

Several parameterizations are popular for the exponential
family. The most general form of the distribution \[
p(x,\theta) = f(x,\theta)e^{g(x,\theta)} \in C^2(\dblr \otimes
\dblr ) \otimes C^2(\dblr \otimes \dblr )\].  The estimators
derived below assume that $f$ and $g$ are separable,
\[ p(x,\theta) = f(x) h(\theta) e^{\alpha(x) \beta(\theta)} \in
C^2(\dblr) \otimes C^2(\dblr ) \otimes C^2(\dblr ) \otimes
C^2(\dblr ) \].

From \[ \int\limits_{x=-\infty}^{x=+\infty} p(x,\theta) dx \:
=1\] we get
\[ \fderiv{\theta}{p(x,\theta)} = 0 = \sderiv{\theta}{p(x,\theta)}d \]
Since the parametrization we have chosen for the exponential
family allows, in the sequel we drop the notation for dependent
variable and denote the derivative with a prime.
\[ \fderiv{\theta}{p(x,\theta)} = \fderiv{\theta}{f h e^{\alpha
\beta}} = h' f e^{\alpha \beta} + f h \alpha \beta' e^{\alpha
\beta} = \bigl( \frac{h'}{h} + \alpha \beta' \bigr) p(x,\theta) \]
which gives
\[\int \fderiv{\theta}{p(x,\theta)} dx = \int \bigl( \frac{h'}{h} + \alpha \beta' \bigr)
p(x,\theta) dx= \frac{h'}{h} \int p(x,\theta) dx \:+\: \beta' \int
\alpha(x) p(x,\theta) dx = \frac{h'}{h}+\beta' E[\alpha(x)] \] so
that \[E[\alpha(x)]=-\frac{h'}{h \beta'}\]. Continuing along
this vein,\begin{gather*} 0 =\int \sderiv{\theta}{p(x,\theta)}
dx = \int \fderiv{\theta}{\bigl(
\frac{h'}{h} + \alpha \beta' \bigr) p(x,\theta) } dx =\\
 \int \bigr(\frac{h''}{h}-\frac{(h')^2}{h^2}+\alpha \beta''
\bigl ) p(x,\theta) + (\frac{h'}{h}+\alpha \beta')
\fderiv{\theta}{p(x,\theta)} \: dx = \\
\int \bigr(\frac{h''}{h}-\frac{(h')^2}{h^2}+\alpha \beta''
\bigl ) p(x,\theta) + (\frac{h'}{h}+\alpha \beta')^2
p(x,\theta) \: dx =
\\ \int \bigr(\frac{h''}{h}-\frac{(h')^2}{h^2}+\alpha \beta'' \bigl )
p(x,\theta) + (\frac{h'}{h}+\alpha \beta')^2 p(x,\theta) \: dx
=
\\ \int \bigr(\frac{h''}{h}-\frac{(h')^2}{h^2}+\alpha \beta'' \bigl )
p(x,\theta) + (\alpha \beta'- E[\alpha(x)]\beta')^2 p(x,\theta)
\: dx
\end{gather*}  Keeping in mind that
 \[ Var[a x ]= E[ (ax-E(ax)^2 ] = a^2 E [ (x-E[x])^2] =a^2 Var[x]
 \]  we get the variance via\[ \bigr(\frac{h''}{h}-\frac{(h')^2}{h^2}+E[\alpha(x)] \beta'' \bigl
)+ Var[\alpha(x) \beta'(\theta)] =
\bigr(\frac{h''}{h}-\frac{(h')^2}{h^2}+E[\alpha(x)] \beta''
\bigl )+ (\beta')^2 Var[\alpha(x)] = 0 \]

The score $U(x)$ is given by \[
U(x)=\pfderiv{\theta}{L(\theta,x)} = \pfderiv{\theta}{\log \:
p(x,\theta)}=\pfderiv{\theta}{\bigr( \log h(\theta) + \log f(x)
+ \alpha(x) \beta(\theta)\bigl) } = \frac{h'}{h} + \alpha
\beta'\] so
\[E[U(x)]=\beta'E[\alpha(x)] + \frac{h'}{h} =0 \].  The Fisher
Information $\mathcal{F}$ is defined \[\mathcal{F}=Var[U(x)]
=Var[ \alpha \beta' + \frac{h'}{h}]= Var[ \alpha \beta'] \] So
from above we have \[Var[U(x)]= Var[ \alpha \beta']
=\bigr(-\frac{h''}{h}+\frac{(h')^2}{h^2}-E[\alpha(x)] \beta''
\bigl )\].  Now differentiating,  \[
\fderiv{\theta}{U(\theta,x)} = \frac{h''}{h} -
\frac{(h')^2}{h^2} + \alpha \beta''\]
\[E[U'(\theta,x)]=\frac{h''}{h} - \frac{(h')^2}{h^2} + E[\alpha]
\beta'' =  \frac{h''}{h} - \frac{(h')^2}{h^2} -\frac{ \beta''
h'}{\beta'}= - Var[U(x)] \].  Note that if we write the
parametrization of the separable exponential family as
\[ p(x,\theta) =  e^{\alpha(x) \beta(\theta)+\log(f(x))+\log(h(\theta))}\]
then,
\[\sderiv{\theta}{\log(h(\theta))}=\fderiv{\theta}{\frac{h'}{h}}=\frac{h''}{h}-\frac{(h')^2}{h^2}\].




%bbcrevisit - this duplicates and has notation clash with material above.
A general form of the exponential distribution
\begin{equation}
\rho(x;\theta) = exp( \frac{x \theta - \xi(\theta) }{\sigma} ) \nu
( x)
\end{equation}
has a log likelihood for a random sample $\{ X_i  \}_{i=1
\hdots N}$ given functionally by
\begin{equation}
\mathcal{L} (\theta) =  \sum\limits_{i=1}^{N} [ X_i  \theta  - \xi
(\theta) + log   ( \nu ( X_i ) ) ]
\end{equation}
The scale parameter $\sigma$ and $\theta$ are orthogonal
parameters in that E [ ] The Generalized Linear model can

$\rho'$ is referred to as a link function in the statistical
literature.  If $\rho'(x)= x\field(1)$ and
$\epsilon=(\epsilon_1, \hdots ,\epsilon_n)$ are iid
$N(\mu,\sigma)$ we have multiple linear regression.  In
classification problems or binomial models the logit
$\rho'(x)=log(x/(1-x))$ link function is used. The logit is
extended to the $k$ category case by \begin{equation}\rho'( x_i
| x_j j \neq i)= log ( \frac{x_i}{1- \sum\limits_{j \neq i}
x_j})\end{equation}. The posterior probability densities
${p_i(?)}$ bbcrevisit (or $p_i$ the probability of observing
class $i$)  of k classes are modeled by linear functions of
the input variables $x_i$.

%\subsection{Multinomial and ordinal Regression}
%When dealing with nominal data, one can consider all classes at
%once via the , or model
%A logistic model for predicting $P(Y | X)$ has a response
%\begin{equation} p(y|x)\equiv p(x) = \frac{1}{1+e^{- x \circ w}
%}\end{equation}. A quick calculation shows that the variance of
%our model is given by $\sigma^{2} = \int p(y|x) (x-\mu)^2 dx$. The
%variance is $p(x)(1+p(x))$

\section{Fitting the GLM}
Iteratively re-weighted least squares (IRLS) is used to for fitting generalized linear models and in finding M-estimators.  The objective function

\begin{equation}
J(\beta^{i+1}) = arg min \sum w_i ( \beta) | y_i - f_i (\beta) |
\end{equation}

is solved iteratively using a Gauss-Newton or Levenberg-Marquardt (LM) algorithm. LM is an iterative technique that finds a local minimum of a function that is expressed as the sum of squares of nonlinear functions. It is a combination of steepest descent and the Gauss-Newton method. When the current solution is far from the minimum the next iterate is in the direction of steepest descent. When the current solution is close to the minimum the next iterate is a Gauss-Newton step.

Linear least-squares estimates can behave badly when the error is not normal.  Outliers can be removed, or accounted for by employing a robust
regression that is less sensitive to outliers than least squares.  M-Estimators were introduced by Huber as a generalization to maximum likelihood estimation.  Instead of trying to minimize the log likelihood

\begin{equation}
L(\theta) = \sum - log ( p(x_i, \theta)
\end{equation}

Huber proposed minimizing

\begin{equation}
M(\theta) = \sum  \rho(x_i, \theta)
\end{equation}

where $\rho$ reduces the effect of outliers. Common loss function are the Huber, and Tukey Bisquare.  For $\rho(x) = x^2$ we have the familiar least squares loss.

M estimators arise from the desire to apply Maximum Likelihood
Estimators to noisy normal data, and to model more general
distributions. They provide a regression that is robust against
outliers in the training set, and allow for modeling of
non-Gaussian processes. When $\rho$ above is a probability
distribution, we are preforming a maximum likelihood
estimation.

The Huber function which is a hybrid $L^2$ $L^1$ norm
\begin{equation}
\rho_\eta(e_i)=\biggl\{\begin{array}{cc}
\frac{e_i^2}{2} & |e_i| \leq \eta \\
  \eta |e_i| - \frac{\eta^2}{2} & |e_i| > \eta \\
\end{array}
\end{equation}
The  Tukey Bisquare estimator is given by
\begin{equation}
g_\eta(e_i)=\biggl\{ \begin{array}{cc} \frac{\eta^2}{6} (
1-[1-\frac{e_i}{\eta}_2]^3) & |e_i| \leq \eta \\
\frac{\eta^2}{6} & |e_i| > \eta \\
\end{array}
\end{equation}

Numerical procedures for doing this calculation are the
Newton-Raphson method [see the section on root finding below ],
and Fisher-Scoring method [ replace $ \frac{\partial^2
\mathcal{L}(\mathbf{\theta})}{\partial \mathbf{\theta} \partial
\mathbf{\theta}^{t} }$ with $E[ \frac{\partial^2
\mathcal{L}(\mathbf{\theta})}{\partial \mathbf{\theta} \partial
\mathbf{\theta}^{t} }  ]$. For high dimensional data, many
models may be fit in an attempt to find the simplest one that
can explain the data.

In the language of statistical learning theory, the choice of a
norm $\rho$ is tantamount to choosing a loss function.
Restricting the admissible functions to the one parameter
family of exponential probability distributions defines the
capacity via a functional form of the law of large numbers.
\cite{Scholkopf B. (2002)}


\section{Feature Subset Selection (FSS)}
The goal of feature selection techniques to to improve the model building process by eliminating features that do not have discriminative power. Algorithms for feature selection either rank features or create subsets of increasing optimality.  FSS should be contrasted with feature extraction techniques such as PCA, LLE, or Laplacian eigenmaps.  The goal of feature extraction is to transform data from a high dimensional space to a low dimensional one while preserving the relevant information.

The statistical approach to feature selection most commonly used is stepwise regression.  Common optimality criteria are FS schemes the Kolmogorov-Smirnov Test ,the t-test, the f-test, the Wilks Lambda Test and Wilcoxon Rank Sum Test.

Feature subset selection (FSS) is the process of determining which measurements will be used for classification.  It's important to distinguish this process from a data dimension reduction process such as PCA which requires all the original measurements to compute the projection. The better FSS algorithms are recursive

Construct a $p x M$ basis matrix $H^{T}$ and transform feature vector $x' = H^{T} x$.

Generalize to $L^{2}$ with smoothing splines

Smoothing spline $RSS(f,\lambda)= \sum\limits_{i=1}^{N} (y_{i}
-f(x_{i}) )^{2} + \lambda \int f''(t)^{2} dt$. where $f \in
C^{2}(\field{R} )$ This is minimized in $L^{2}$ the first term
measuring closeness of fit, and the second term penalizes
curvature. $\lambda \rightarrow 0$ gives any function
interpolating the data points ${x_i}_{i  \in {1, ... N} } $ an
$\lambda \rightarrow \infty$ constrains $f$ to be linear.


\section{Longitudinal Data Analysis} Longitudinal data analysis
is the observation of multiple subjects over repeated
intervals. Binary repeated responses are typically modelled
with a marginal or random effects model, which will be made
precise below. Marginal Models are a generalization of the GLM
presented above for correlated data.  Here, the correlation is
inter subject across time.  Statistical analysis of
longitudinal data must take into account that serial
observations of a subject are likely to be correlated, time may
be an explanatory variable, and that missing response data my
induce a bias in the results.

Let ${X_{ij}}$ be time varying or fixed covariates for the
binary response ${Y_{ij}}$ of subject $i \in {1,...n}$ at time
intervals $j \in {t_1,...t_m}$. By convention $X_{ij} \in
\field{R} x \field{R^p}$ where the first dimension is the
intercept. The marginal model is; $logit (E(Y_{ij} | X_{ij}) )
= X_{ij}^{\dagger} \beta$ and enforces the assumption that the
relationship between the covariates and the response is the
same for all subjects. Recall that for a binary response,
$E(Y_{ij} | X_{ij}) = P(Y_{ij}=1 | X_{ij})$.  The random
effects model takes into account that the relationship between
the covariates and response varies between subjects; $logit
(P(Y_{ij}=1 | X_{ij}) ) = X_{ij}^{\dagger} \beta_i$ If it is
know that only a subset of the covariates are involved in the
inter-subject variability, we can set $\beta_i= \beta +
\beta_i$ and write $logit (P(Y_{ij}=1 | X_{ij}, \beta_i) ) =
X_{ij}^{\dagger} \beta + O X_{ij} \beta_i$ Where the kernel of
$O : \dblr^n \rightarrow \dblr^{n'}$ is the span of the
covariates that do not change between subjects.  If $\lambda_i
=_d N(0,\sigma)$ then the difference in the parameter vectors
$\beta$ in the two models differ according to $\sigma$.

The GEE method of fitting the marginal model is described in:
\cite{Liang, K-Y and Zeger, S. L.(1986)}

The Survival Analysis is a form of longitudinal analysis that
takes into consideration the amount of time an observation is
made on a subject.

GLM's can be used to fit discrete longitudinal hazard models
derived from survival analysis, see  \cite{Prentice and
Gloeckler (1978)}.   \cite{Meyer, B.D. (1990)} generalized that
approach to account for an unobserved subject heterogeneity.

\cite{Holmen, M (2005)} applied the hazard model of
\cite{Prentice, R. and L. Gloeckler (1978)} to the takeover
hazard of large firms.  A negative relationship between dual
class ownership and value is empirically known, and that
relationship can be explained by the lower takeover probability
of the dual class firms.  Dual class entities had a higher risk
for takeover, but the hazard is lower since these firms use the
dual class structure to change the capital structure in a way
that allows the controlling shareholders to remain in control
by reducing firm value.

The proportional hazards model can be discretized, but it is
important to identify whether the process is truly a discrete
process.  In that case the link function should be the logit as
the Marginal Model above specifies, rather than the log-log
function of the discretized proportional model.  The difference
is the modelling of a probability transition in the former case
versus a rate for the latter case.

Variable selection techniques for longitudinal data are
relatively limited and most seem to rely on Wald type tests.
Wald tests to include a variable are based on already computed
maximum likelihood values. The Rao score test is used to
include a covariate in the model building process.  The Wald
test calculates
\[z^2=\frac{\widehat{\beta}}{stderr}=_d  \chi^2\]

The likelihood ratio statistic for comparing two models $L_0
\in L_1$ \[-2 \frac{L_0}{L_1} =_d \chi^2\] is useful for
backward stepwise variable subset selection. The degrees of
freedom of the of the statistic is equal to the difference in
dimension of the two models.

\section{Discretization \& Sheppard's Correction}
W. Sheppard (1898) Derived an approximate relationship
between the moments of a continuous distribution and it's
discrete approximation. This provides a transformation to
statistical estimators that correct for the binning of
continuous data.  As the scale at which datum are collected is
increased, the variance of an estimate can become biased.

It is important to assess  bias caused by grouping and to correct it if necessary.
The  bias of the approximate maximum likelihood estimator where
observations are approximated by interval midpoints $O(w^2)$, where $w$ is the bin width. A Sheppards correction can be used to reduce the bias to order $O(w^3)$,

Signal processing engineers often have to deal with such a
quantization effect when designing finite precision systems,
image processing being a particularly relevant example. The
engineering community typically models the quantization noise
$Q=[X]-X$, where $[X]$ is the quantized realization of $X$. One
might be tempted to apply a Sheppard's correction to the
moments of the quantized data, thinking that $Var(X)<Var([X])$
but it is possible to construct examples where $Q$ and $[X]$
are independent, or where $Cov(X,Q)$ is such that
$Var(X)>Var([X])$.

Shepard's correction is limited in that is doesn't apply to the
first moment, and the frequencies of the first and last bins
need to be low.

Expand $p(x;\theta)$ in a Taylor series and substitute in the
Maximum Likelihood equations. \cite{Lindley, D. V. (1950)}

Suppose we have n realizations of iid RV's ${X_1, \hdots ,
X_n}$ and the data is collected on a discrete grid on the range
of $X$ $Ran(X)=\{[y_i-d_i/2,y_i+d_i/2]\}_{i=1}^{i=m}$ where the
intervals are centered on the location where a measurement. The
realized values ${y_1, \hdots , y_m}$ have probabilities
$p_i=\int\limits_{y_i - d_i /2}^{y_i+d_i /2} p(x;\theta) \;\;
dx$ Expanding $p(x;\theta)$ in a Taylor series about $y$,
$p(x;\theta)= \sum\limits_{i=0}^{\infty} \frac{p^{(i)}(y) }{i!}
(x-y)^i$.


\section{Multidimensional Scaling}
Multidimensional scaling (MDS) is an alternative to factor analysis. The aim of MDS and factor analysis is to detect meaningful underlying dimensions that explain similarities or dissimilarities data points. In factor analysis, the similarities between points are expressed via the correlation matrix. With MDS any kind of similarity or dissimilarity matrix may be used.

Given $n$ observations ${x_i}_{i=1}^{n} \in \dblr^k$ and $n^2$
distances $d_{ij}$ between them, MDS looks for $n$ points
${\xi_i}_{i=1}^{n}$ in $dblr^l : l<k$ that preserve the
distance relations. When a metric $\rho()$ exists for the similarity measure, gradient
descent is used to minimize the MDS functional $S(\xi_1, \ldots
, \xi_l)=\biggl( \sum_{i \neq j}
d_{ij}-||\xi_i-\xi_j||_{\rho}\biggr)^\frac{1}{2}$.

\section{Principal Components} For a data set $\textbf{X} \in
M_{(N,m)}(\mathbb{R}) = { x_1, x_2, \ldots x_N | x_i \in
\mathbb{R}^m } $, the first k principal components provided the
best k dimensional linear approximation to that data set.
Formally, we model the data via $f(\theta) = \mu + \textbf{V}_k
\theta | \mu \in \mathbb{R}^m, V_k \in O_{m,k}(\mathbb{R}),
\theta \in \mathbb{R}^k$ so $f(\theta)$ is an affine hyperplane
in $\mathbb{R}^m$

\section{Evaluating classifier performance} Multi-class problems
can be treated simultaneously or broken in to a sequence of two
class problems.  Cross validation is used both for classifier parameter tuning
and for feature subset selection.  Student-t and ANOVA can be used to evaluate the performance of classifiers against one another.  The Student-t test compares
two classifiers, while the ANOVA test can compare multiple
classifiers against one another.  Confusion matrices and ROC graphs are commonly employed visualization tools for assessing classifier performance. The
rows of a confusion matrix add to the total population for each
class, and the columns represent the predicted class.  An ROC
curve plots the TP rate against the FP rate. Often a curve in
ROC space is drawn using classifier parameters for tuning
purposes.

\begin{table}[h]
\begin{tabular}{|c|c|}
  \hline
  % after \\: \hline or \cline{col1-col2} \cline{col3-col4} ...
  TN &  FP \\
  \hline
  FN & TP \\
  \hline
\end{tabular}
\caption{Two class confusion matrix where the proportions are
specified}
\end{table}

Common performance metrics for the two class problem are
sensitivity (TP), specificity (TN), precision (the proportion
of predicted cases within a class that were correct), and
accuracy (the overall proportion of correct predictions). These
metric can be extended to more than two classes by defining
$A=tr ( C ) / || C ||_{L^\infty}$ where $C$ is the confusion
matrix. TP, FN, FP, TN are proportions defined for the two
class problem.

\section{Covariance Matrix Estimation } For numerical stability
in regression algorithms, the covariance matrix needs to be
positive definite.  An well conditioned estimator for the
covariance matrix of a process can be obtained by mixing the
sample covariance with the identity matrix.  This is a linear
shrinkage estimator based on a modified Frobenius norm for $A
\in M_{mn}$  \begin{equation} ||\mathbf{A}||_{\cal{F}}= \sqrt{
\frac{tr (A A^t)}{n}} \end{equation}  Without loss of
generality, set $\mu =0$ and let $\widehat{\Sigma} = \alpha
\mathbb{I} + \beta \mathbf{S}$ where $\mbf{S}=\frac{\mbf{X}^T
\mbf{X}}{n}$ is the sample covariance.  We seek to minimize $E(
||\widehat{\Sigma} - \Sigma||^2)$, but since we don't know the
true population covariance matrix, we have to form an
approximation.

%Nonlinear Shrinkage Estimation of Large-Dimensional Covariance Matrices
%Olivier Ledoit
%Many statistical applications require an estimate of a covariance matrix and/or its inverse.
%When the matrix dimension is large compared to the sample size, which happens
%frequently, the sample covariance matrix is known to perform poorly and may suffer from
%ill-conditioning. There already exists an extensive literature concerning improved estimators
%in such situations. In the absence of further knowledge about the structure of the
%true covariance matrix, the most successful approach so far, arguably, has been shrinkage
%estimation. Shrinking the sample covariance matrix to a multiple of the identity, by taking
%a weighted average of the two, turns out to be equivalent to linearly shrinking the sample
%eigenvalues to their grand mean, while retaining the sample eigenvectors
%
%Covariance Estimation:
%The GLM and Regularization Perspectives
%Mohsen Pourahmadi
%Department of Statistics
%Texas A&M University
%The p  p covariance matrix  of a random vector Y = (y1;    ; yp)0 with as many as p(p+1)
%2constrained parameters plays a central role in virtually all of classical multivariate statistics(Anderson, 2003), time series analysis (Box et al. 1994), spatial data analysis (Cressie,
%1993), variance components and longitudinal data analysis (Searle et al. 1992; Diggle et
%al. 2002), and in the modern and rapidly growing area of statistical and machine learning
%dealing with massive and high-dimensional data (Hastie, Tibshirani and Friedman, 2009).
%More specically, principal component analysis, factor analysis, classication and cluster
%analysis, inference about the means and regression coecients, prediction and Kriging, and
%analysis of conditional independence in graphical models typically, require an estimate of a
%covariance matrix or its inverse. It is generally recognized that the two major challenges
%in covariance estimation are the positive-deniteness constraint and the high-dimensionality
%where the number of parameters grows quadratically in p. In this survey, we point out
%that the latter challenge is virtually eliminated by reducing covariance estimation to that of
%solving a series of penalized least-squares regression problems. 