% ----------------------------------------------------------------
% AMS-LaTeX Paper ************************************************
% **** -----------------------------------------------------------
\documentclass[10pt]{amsart}
%\documentclass{report}
\usepackage{graphicx}
% ----------------------------------------------------------------
\vfuzz2pt % Don't report over-full v-boxes if over-edge is small
\hfuzz2pt % Don't report over-full h-boxes if over-edge is small
% THEOREMS -------------------------------------------------------
\newtheorem{thm}{Theorem}[section]
\newtheorem{cor}[thm]{Corollary}
\newtheorem{lem}[thm]{Lemma}
\newtheorem{prop}[thm]{Proposition}
\theoremstyle{definition}
\newtheorem{defn}[thm]{Definition}
\theoremstyle{remark}
\newtheorem{rem}[thm]{Remark}
\numberwithin{equation}{section}
% MATH -----------------------------------------------------------
\newcommand{\norm}[1]{\left\Vert#1\right\Vert}
\newcommand{\abs}[1]{\left\vert#1\right\vert}
\newcommand{\set}[1]{\left\{#1\right\}}
\newcommand{\Real}{\mathbb R}
\newcommand{\eps}{\varepsilon}
\newcommand{\To}{\longrightarrow}
\newcommand{\BX}{\mathbf{B}(X)}
\newcommand{\A}{\mathcal{A}}

\newlength{\defbaselineskip}
\setlength{\defbaselineskip}{\baselineskip}
\newcommand{\setlinespacing}[1]%
           {\setlength{\baselineskip}{#1 \defbaselineskip}}
\newcommand{\doublespacing}{\setlength{\baselineskip}%
                           {2.0 \defbaselineskip}}
\newcommand{\singlespacing}{\setlength{\baselineskip}{\defbaselineskip}}
%Example of usage

% ----------------------------------------------------------------
\begin{document}
\setlength{\baselineskip}{1.6 \defbaselineskip}
\title{Proposal to Develop Semi Automated Her2 FISH Machine Vision Software}%
\author{Bruce B. Campbell}%
{Wavescholar Consulting, LLC.}
\email{bruce@wavescholar.com}%

% ----------------------------------------------------------------
\begin{abstract}
Her2 assessment in breast cancer provides important predictive and
patient management information.  Biologic mechanisms underlying
Her2 involvement in breast cancer is currently evaluated in the
clinical setting via fluorescence in situ hybridization (FISH) for
gene amplification and immunohistochemistry (IHC) for protein
over-expression on the cell membrane.  IHC is a relatively
inexpensive test and can be done in most clinical pathology labs,
but grading is qualitative and subject to inter-observer variation
that may have clinical implications. FISH analysis shows good
concordance with high grade IHC results and can resolve
inconclusive IHC results, but is expensive to implement and more
laborious for the pathologist to score.  This document seeks to
make a case that machine vision software can be developed to
reduce the labor burden in grading FISH.  It is speculated that
the labor barrier may a bigger issue in adaptation rather than the
hardware required to implement the Her2 FISH assay.  A machine
vision system may alleviate that cost concern and provide valuable
clinical benefit.

\end{abstract}

\maketitle

\tableofcontents

\section{The importance and challenges of accurate assessment of Her2
over-expression / amplification in metastatic breast cancer} In
20-30\% of breast cancer cases the human epidermal growth factor
receptor type 2 (HER-2) gene is over-expressed.  The
over-expression of Her-2 is generally a result of gene
amplification where multiple copies of the gene are present. HER2
status is predictive of disease free status post chemotherapy, and
is use to identify patients most likely to benefit from Herceptin
therapy. Current best practice in assessing HER2 status in
metastatic breast cancer is to test for amplification using IHC
and to confirm ambiguous (2+) cases with FISH.  IHC scores of 0,1
are assumed to be FISH negative and IHC scores of 3 are assumed to
be FISH positive.  There is good statistical concordance between
the 0,1,3 IHC and FISH scores, but it is not 100\%.  The
subjective nature of IHC grading allows for the possibility that
some IHC grade 1 cases would be HER2 positive by FISH. For
background information see \cite{2,3}.

\section{Automated / Semi Automated Pathology Background}
The job of the clinical pathologist can be labor intensive.
Attempts have been made to alleviate some of the visual burden
through digital capture and display, semi-automated, and fully
automated systems. Semi-automated pathology would include digital
capture and display and functionality to improve the ability to
grade via simple color segmentation or counting algorithms.  Fully
automated pathology would produce whole slide grades based.

There is keen interest interest and due skepticism regarding fully
automated pathology systems. The track record of vision software
that accurately reproduces path scores is not good frankly.
Challenges in developing such systems include;
\begin{itemize}
    \item The visual complexity and high degree of variation in biological images
    poses a challenge to developing accurate tissue segmentation
    software.
    \item The difficulty and expense in obtaining sufficient
    training and test data for developing grading software.
    \item Accommodating in software the color and morphometric variation due to
    operator/equipment induced variation in staining, tissue preparation, and image capture.
    \item The propensity for pathologists to disagree.
\end{itemize}

It has been demonstrated that whole slide digital scanning and
display does not affect the ability to make reliable clinical
decisions \cite{1}.  There is an element of luck in being able to
migrate a clinical test from digital pathology to automated
pathology. The stain,tissue morphology and psychophysics involved
in scoring must be amenable to the capture hardware, imaging
software and statistical tools at hand.  Obtaining sufficient
training data for tissue object and whole slide models is also a
risk point in developing automated applications.

\section{Machine Vision in Pathology}
Automated pathology workflow
\begin{itemize}
    \item scan, annotate, store
    \item color / morphological object segmentation
    \begin{itemize}
\item pixel classification \item object classification \item image
tile classification
\end{itemize}
    \item whole slide classification
\end{itemize}
The last step is typically the hardest.  An ordinal scoring model
based on expert data maps the object data onto a path score.
Applications involving simple visual measurements such as
vacuolation or thickening of a membrane are easier for machine
vision scoring models.  Tasks like low grade necrosis or
inflammation that require subtle discrimination of color or
texture are harder. Tasks like detecting mitosis that require
finding one bad object among many good are very hard for machine
vision applications.

\section{Automated scoring of HER2 FISH digital images}
PathVysion (Vysis Inc.) and Inform (Ventana Medical Systems) two
commercially available tests to the evaluation of HER2 status.
Inform is a single probe system counterstained with DAPI or PI.
The PathVysion test is a two probe HER-2/CEP17 test counterstained
with DAPI.  Soring of the PathVysion test is done by locating 60
tumor cells and determining the HER-2/CEP17 signal ratio.  A ratio
$\succeq 2.0$ is the cutoff for amplification. For some window
around the cutoff $( \pm -.2 )$, an additional 60 nuclei are
counted. The range and distribution of 2,502 HER-2/CEP17 ratios in
\cite{3} shows that the there are a lot of spots to count for an
accurate assessment of an amplified sample.

Scoring of HER2 FISH samples is an ideal candidate for
semi-automated pathology.  The HER-2/CEP17 ratio is a quantitative
score, so some of the difficulties outlined above are alleviated.
Once tumor cells are located, scoring becomes a moderate machine
vision task. If an H\&E reference slide is available it can be
digitally registered with the FISH image, and  a pathologist
selected region of interest (ROI) on an H\&E area containing tumor
cells is easily mapped onto the FISH image.

Variance in sample preparation via staining and tissue thickness
will often produce variances in luminance that can make machine
vision applications more challenging.  By removing the luminance
information and separating colors in a perceptual color space, a
more robust segmentation can be achieved.  FISH images are good
candidates for segmenting in chromaticity space, because the
colors are determined by the markers and optical filters - which
tend to be stable colorimetrically.

Cell segmentation is accomplished with moderate ease on FISH image
using common imaging tools. Once cells are segmented, they must be
classified as tumor cell or not. Classification can be
accomplished two ways
\begin{itemize} \item morphologically with characteristics
provided by the test manufacturer \item via labelled cells
obtained from the pathologist
\end{itemize}
Which method to choose depends on the nature of the FISH test.
Generally is will be a size condition, but for some applications
the criteria will be more subjective - ie 'a thready appearance to
DAPI stain'.  For the latter case, a hand made wavelet filter
would be more appropriate.  This is part is again where the matter
of luck comes in - with labelled data an classifier may be able to
easily discriminate tumor from non tumor based on simple
morphological measures.  Filtering is more of an art compared
classification using morphological characteristics.

The results of a segmentation in the Lab perceptual colorspace is
below.  A transform can be applied in the direction of the cell
stain color to obtain a quantitative measure of the stain that the
pathologist sees for the Inexpensive and open source imaging
toolkits are available from Intel to implement this in C++ for
high volume applications.  The masks presented would be used to
collect data from the original image for classification as
cancerous or not.

\newpage
\includegraphics[width=8.0cm,height=8.0cm]{c://bruce//HER2Fish//FISH_WORK//FISH_IMAGES//HER2//NSABP//B31-1677A108.jpg}
\includegraphics[width=8.0cm,height=8.0cm]{c://bruce//HER2Fish//FISH_WORK//FISH_IMAGES//HER2//NSABP//B31-1677A108_abConvexHull.jpg}
\includegraphics[width=8.0cm,height=8.0cm]{c://bruce//HER2Fish//FISH_WORK//FISH_IMAGES//HER2//NSABP//B31-1677A108_RGB_LabelImg.jpg}
\includegraphics[width=8.0cm,height=8.0cm]{c://bruce//HER2Fish//FISH_WORK//FISH_IMAGES//HER2//NSABP//B31-1677A108_SampleChromas.jpg}
\newpage
A positive expression of HER2 amplification.\newline
\includegraphics[width=8.0cm,height=8.0cm]{c://bruce//HER2Fish//FISH_WORK//FISH_IMAGES//HER2//NSABP//her-2positive.jpg}
\includegraphics[width=8.0cm,height=8.0cm]{c://bruce//HER2Fish//FISH_WORK//FISH_IMAGES//HER2//NSABP//her-2positive_abConvexHull.jpg}
\includegraphics[width=8.0cm,height=8.0cm]{c://bruce//HER2Fish//FISH_WORK//FISH_IMAGES//HER2//NSABP//her-2positive_RGB_LabelImg.jpg}
\includegraphics[width=8.0cm,height=8.0cm]{c://bruce//HER2Fish//FISH_WORK//FISH_IMAGES//HER2//NSABP//her-2positive_SampleChromas.jpg}
\newpage

With a mask of cancer cells a model for generating the HER-2/CEP17
ratio can be made.  This again can be done with labelled
pathologist data, or by modelling based on specifications from the
FISH application.  Spot counting is easy up to a point, in the
high end of the range of the distribution of ratio the spots start
to coalesce. Looking at the positive expression on the previous
page at 600\% zoom, one can see the some of signals starting to
merge. Dealing with this will require a robust statistical
approach to handle the transition from an expression where simple
spot counting can be done, to one that must measure the signal
strength of the amplification to estimate the ratio. Compounding
this is that some FISH signals can be ghosts of real ones and must
be eliminated from the count. What makes ghosts tractable for
moderate amplification is that they will be an affine transform
away from the real signal and frequency based filtering methods
can be used to detect them.  High amplification will present a
problem to ghost detection, so the model from moderate to high
amplification needs to handled carefully.





%
%
%\begin{center}
%\begin{tabular}{cc}
%
%\includegraphics[width=6.0cm,height=6.0cm]{c://bruce//HER2Fish//FISH_WORK//FISH_IMAGES//HER2//NSABP//B31-1072A1.jpg}
%&
%\includegraphics[width=6.0cm,height=6.0cm]{c://bruce//HER2Fish//FISH_WORK//FISH_IMAGES//HER2//NSABP//FISH_convexhull_Chroma_kmenasSegment__B31-1072A1.jpg}
%\\
%\includegraphics[width=6.0cm,height=6.0cm]{c://bruce//HER2Fish//FISH_WORK//FISH_IMAGES//HER2//NSABP//B31-1072A45.jpg}
%&
%\includegraphics[width=6.0cm,height=6.0cm]{c://bruce//HER2Fish//FISH_WORK//FISH_IMAGES//HER2//NSABP//FISH_convexhull_Chroma_kmenasSegment__B31-1072A45.jpg}
%\\
%\includegraphics[width=6.0cm,height=6.0cm]{c://bruce//HER2Fish//FISH_WORK//FISH_IMAGES//HER2//NSABP//B31-1443A201.jpg}
%&
%\includegraphics[width=6.0cm,height=6.0cm]{c://bruce//HER2Fish//FISH_WORK//FISH_IMAGES//HER2//NSABP//FISH_convexhull_Chroma_kmenasSegment__B31-1443A201.jpg}
%\end{tabular}
%\end{center}
%




\begin{thebibliography}{12}
\bibliographystyle{amsplain}

\bibitem{1} Primary histologic diagnosis using automated whole slide
imaging: a validation study. John R Gilbertson, Jonhan Ho, Leslie
Anthony, Drazen M Jukic, Yukako Yagi and Anil V Parwani
 BMC Clinical Pathology 2006, 6:4

\bibitem{2} Comparison of HER2 Status by Fluorescence in Situ
Hybridization and Immunohistochemistry to Predict Benefit From
Dose Escalation of Adjuvant Doxorubicin- Based Therapy in
Node-Positive Breast Cancer Patients.  Lynn G. Dressler, Donald A.
Berry, Gloria Broadwater, David Cowan, Kelly Cox, Stephanie
Griffin, Ashley Miller, Jessica Tse, Debra Novotny, Diane L.
Persons, Maurice Barcos, I. Craig Henderson, Edison T. Liu, Ann
Thor, Dan Budman, Hy Muss, Larry Norton, and Daniel F. Hayes. J
Clin Oncol 23:4287-4297.

\bibitem{3} Diagnostic Evaluation of HER-2 as aMolecularTarget: An
Assessment of Accuracy and Reproducibility of Laboratory Testing
in Large, Prospective, Randomized Clinical Trials. Michael F.
Press, Guido Sauter, Leslie Bernstein Ivonne E.Villalobos, Martina
Mirlacher, Jian-Yuan Zhou, RoobaWardeh, Yong-Tian Li, Roberta
Guzman, Yanling Ma, Jane Sullivan-Halley, Angela Santiago, Jinha
M. Park, Alessandro Riva, and Dennis J.Slamon.  Clin Cancer Res
2005;11(18) September 15, 2005

\end{thebibliography}




% ----------------------------------------------------------------
\end{document}
% ----------------------------------------------------------------
